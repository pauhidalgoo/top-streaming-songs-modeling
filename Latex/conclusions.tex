\section{Conclusions}

Al llarg d'aquest projecte s'han utilitzat moltes eines diferents per transformar, analitzar, modelar i exprimir la base de dades amb l'objectiu de poder extreure informació útil i d'interès, aportant als membres d'aquest grup un coneixement ampli en múltiples aspectes de l'estadística, l'anàlisi i la ciència de dades.

% Conclusions D3
Les tècniques de \textit{clustering} incloent els mètodes més tradicionals com ara el \textit{K-Means}, \textit{Hierarchical}, \textit{Cure} (entre d'altres) o inclús els mètodes basats en agrupacions de sèries temporals amb els seus respectius \textit{profilings} ens han permés agrupar les cançons en unes classes i posteriorment etiquetar-les .

Unes altres tècniques que ens van ajudar a comprendre millor les nostres classes d'aquests clústers van ser utilitzar semàfors i \textit{TLPs} els quals de forma molt senzilla i visual explicaven i describien cada classe. 

% Conclusions D4

Pel que fa a les conclusions extretes del \textit{textual analysis} podem confirmar que els mètodes estadístics utilitzats com ara l'\textit{LSA} i \textit{LDA} ens han permès comprendre millor les distribucions de les cançons basant-nos simplement en la seva lletra. En aquestes, s'ha observat certa relació entre les lletres de les cançons i el seu gènere, així com evidentment els temes dels quals tracten.

A banda d'això s'han pogut crear unes aplicacions reals com ara el generador de \textit{playlists} o el predictor de gènere que són uns productes els quals podria utilitzar tant un usuari de la plataforma com la mateixa empresa per crear llistes de reproduccions basades en els gustos personals o categoritzar una nova cançó en un gènere sense la necessitat directa d'etiquetar manualment aquella cançó.

Utilitzant les dades \textit{geoespacials}, s'ha pogut observar i modelar diverses característiques músicals en funció del país o la regió del món. Aquests canvis s'han pogut interpretar i relacionar amb la cultura i les característiques de cada punt del món. A més, utilitzat també noves dades, s'ha vist que hi ha certa similitud entre les cançons més escoltades d'un país i les creades pels artistes d'aquell, i s'han pogut observar diversos patrons de consum dels usuaris de Spotify. També s'han pogut realitzar visualitzacions interactives, que aporten cert valor als usuaris, i s'han comparat amb els resultats de l'anàlisi textual.

En general, aquest treball ens ha permès obtenir informació profunda i interessant sobre la nostra base de dades, consistent en informació diversa sobre les cançons del top de Spotify. Amb els mètodes aplicats, s'han pogut esbrinar tota mena de relacions: el país amb l'energia, la lletra d'una cançó amb el gènere, la predicció de si una cançó és explícita... La informació de la qual disposàvem inicialment s'ha complementat amb altres dades relacionades per poder extreure encara més coneixement. S'han espremut les dades al màxim, obtenint relacions inicialment ocultes i analitzant-la completament, però alhora s'ha combinat aquesta tasca amb l'objectiu d'aportar valor hipotèticament a una empresa, sigui la mateixa distribuïdora de música com als artistes. Amb tot, aquest treball ha presentat exemples interessants de tots els apartats, des del preprocessament necessari fins a l'anàlisi textual i geoespacial, i es pot concloure que s'ha sigut capaç d'analitzar completament el top 40 setmanal de Spotify, oferint una visió integral i detallada.

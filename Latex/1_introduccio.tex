\section{Introducció}

Des de l'arribada de les plataformes de streaming, la música s'ha convertit en un element molt present en el dia a dia d'una gran quantitat de persones. En aquest treball, es buscarà analitzar els hàbits de consum a partir de les 50 cançons més escoltades cada setmana.
Les dades amb les que es treballarà provenen de Spotify, una de les empreses amb més usuaris en el sector, i van ser recollides entre el 2017 i el quart mes del 2021.

A partir d'aquesta informació, s'intentarà esbrinar quines són les tendències a les quals evoluciona la música més popular, quins trets tenen en comú les cançons més exitoses i quins estils trobem, entre d'altres.

Durant el transcurs d'aquest projecte s'usaran tècniques de preprocessament i de clústering avançades, així com eines d'anàlisi textual i geoespacial per tal d'extreure el màxim d'informació d'aquestes dades.